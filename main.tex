%%%%%%%%%%%%%
% % % % % % % % % % % % % % % % % % % % % % % % % % % % % % % % % %
%\documentclass[runningheads]{llncs}
%\documentclass[10pt,letterpaper,twocolumn]{article}
\documentclass{sig-alternate}


% packages
\usepackage{xspace}
\usepackage{ifthen}
\usepackage{amsbsy}
\usepackage{amssymb}
\usepackage{balance}
\usepackage{booktabs}
\usepackage{graphicx}
\usepackage{multirow}
\usepackage{needspace}
\usepackage{microtype}
\usepackage{bold-extra}
\usepackage{subfigure}
\usepackage{wrapfig}


% constants
\newcommand{\Title}{Pitfalls and advices on code profiling}
\newcommand{\TitleShort}{\Title}
\newcommand{\Authors}{Alexandre Bergel, Vanessa Pe\~na, Juan Pablo Sandoval}
\newcommand{\AuthorsShort}{A. Bergel, V. Pe\~na, J.P. Sandoval}

% references
\usepackage[colorlinks]{hyperref}
\usepackage[all]{hypcap}
\setcounter{tocdepth}{2}
\hypersetup{
	colorlinks=true,
	urlcolor=black,
	linkcolor=black,
	citecolor=black,
	plainpages=false,
	bookmarksopen=true,
	pdfauthor={\Authors},
	pdftitle={\Title}}

\def\chapterautorefname{Chapter}
\def\appendixautorefname{Appendix}
\def\sectionautorefname{Section}
\def\subsectionautorefname{Section}
\def\figureautorefname{Figure}
\def\tableautorefname{Table}
\def\listingautorefname{Listing}

% source code
\usepackage{xcolor}
\usepackage{textcomp}
\usepackage{listings}
\definecolor{source}{gray}{0.9}
\lstset{
	language={},
	% characters
	tabsize=3,
	upquote=true,
	escapechar={!},
	keepspaces=true,
	breaklines=true,
	alsoletter={\#:},
	breakautoindent=true,
	columns=fullflexible,
	showstringspaces=false,
	basicstyle=\footnotesize\sffamily,
	% background
	frame=single,
    framerule=0pt,
	backgroundcolor=\color{source},
	% numbering
	numbersep=5pt,
	numberstyle=\tiny,
	numberfirstline=true,
	% captioning
	captionpos=b,
	% formatting (html)
	moredelim=[is][\textbf]{<b>}{</b>},
	moredelim=[is][\textit]{<i>}{</i>},
	moredelim=[is][\color{red}\uwave]{<u>}{</u>},
	moredelim=[is][\color{red}\sout]{<del>}{</del>},
	moredelim=[is][\color{blue}\underline]{<ins>}{</ins>}}
\newcommand{\ct}{\lstinline[backgroundcolor=\color{white},basicstyle=\footnotesize\ttfamily]}
\newcommand{\lct}[1]{{\small\tt #1}}

% tikz
% \usepackage{tikz}
% \usetikzlibrary{matrix}
% \usetikzlibrary{arrows}
% \usetikzlibrary{external}
% \usetikzlibrary{positioning}
% \usetikzlibrary{shapes.multipart}
% 
% \tikzset{
% 	every picture/.style={semithick},
% 	every text node part/.style={align=center}}

% proof-reading
\usepackage{xcolor}
\usepackage[normalem]{ulem}
\newcommand{\ra}{$\rightarrow$}
\newcommand{\ugh}[1]{\textcolor{red}{\uwave{#1}}} % please rephrase
\newcommand{\ins}[1]{\textcolor{blue}{\uline{#1}}} % please insert
\newcommand{\del}[1]{\textcolor{red}{\sout{#1}}} % please delete
\newcommand{\chg}[2]{\textcolor{red}{\sout{#1}}{\ra}\textcolor{blue}{\uline{#2}}} % please change
\newcommand{\chk}[1]{\textcolor{ForestGreen}{#1}} % changed, please check

% comments \nb{label}{color}{text}
\newboolean{showcomments}
\setboolean{showcomments}{true}
\ifthenelse{\boolean{showcomments}}
	{\newcommand{\nb}[3]{
		{\colorbox{#2}{\bfseries\sffamily\scriptsize\textcolor{white}{#1}}}
		{\textcolor{#2}{\sf\small$\blacktriangleright$\textit{#3}$\blacktriangleleft$}}}
	 \newcommand{\version}{\emph{\scriptsize$-$Id$-$}}}
	{\newcommand{\nb}[2]{}
	 \newcommand{\version}{}}
\newcommand{\rev}[2]{\nb{Reviewer #1}{red}{#2}}
\newcommand{\ab}[1]{\nb{Alexandre}{blue}{#1}}
\newcommand{\vp}[1]{\nb{Vanessa}{orange}{#1}}

% graphics: \fig{position}{percentage-width}{filename}{caption}
\DeclareGraphicsExtensions{.png,.jpg,.pdf,.eps,.gif}
\graphicspath{{figures/}}
\newcommand{\fig}[4]{
	\begin{figure}[#1]
		\centering
		\includegraphics[width=#2\textwidth]{#3}
		\caption{\label{fig:#3}#4}
	\end{figure}}

\newcommand{\largefig}[4]{
	\begin{figure*}[#1]
		\centering
		\includegraphics[width=#2\textwidth]{#3}
		\caption{\label{fig:#3}#4}
	\end{figure*}}
	
\newcommand{\wrapfig}[5]{	
\begin{wrapfigure}{#1}{#2\textwidth}
  \begin{center}
    \includegraphics[width=#3\textwidth]{#4}
  \end{center}
  \caption{\label{fig:#4}#5}
\end{wrapfigure}}

% abbreviations
\newcommand{\ie}{\emph{i.e.,}\xspace}
\newcommand{\eg}{\emph{e.g.,}\xspace}
\newcommand{\etc}{\emph{etc.}\xspace}
\newcommand{\etal}{\emph{et al.}\xspace}

% lists
\newenvironment{bullets}[0]
	{\begin{itemize}}
	{\end{itemize}}

\newcommand{\seclabel}[1]{\label{sec:#1}}
\newcommand{\secref}[1]{Section~\ref{sec:#1}}
\newcommand{\figlabel}[1]{\label{fig:#1}}
\newcommand{\figref}[1]{Figure~\ref{fig:#1}}
\newcommand{\tablabel}[1]{\label{tag:#1}}
\newcommand{\tabref}[1]{Table~\ref{fig:#1}}


%Specialized macros
\newcommand{\hapao}{Hapao\xspace}
\newcommand{\Hapao}{Hapao\xspace}
\pagenumbering{arabic}

\begin{document}

\title{\Title}
%\titlerunning{\TitleShort}

\author{\Authors\\[3mm]
Department of Computer Science (DCC)\\ University of Chile, Santiago, Chile\\[1 ex]
} 
%\authorrunning{\AuthorsShort}

\maketitle

%\emph{This paper makes use of colored figures. Though colors are not mandatory for full understanding, we recommend  the use of a colored printout.}

\begin{abstract}
%	What's the problem.
Code profilers estimate the amount of time spent in each method by regularly sampling the method call stack. 
One of the main advantage of execution sampling is the relatively low overhead it incurs. However, execution sampling is fairly inaccurate 

%	Why is the problem a problem?

%	What's the surprising idea?
We first 

%	What's the consequence?

\end{abstract}

\section{Introduction}\seclabel{introduction}

The research questions addressed in this paper are:
\begin{itemize}
\item[A -] \emph{Can the distribution of the CPU consumption access objects and messages be accurately determined?}
\item[B-] \emph{}
\end{itemize}


%: % % % % % % % % % % % % % % % % % % % % % % % % % % % % % % % % %
\section{Code Execution Sampling}

\paragraph{Execution sampling}
Execution sampling approximates the time spent in an application's methods by periodically stopping a program and recording the collection of methods being executed. The advantages of execution sampling are multiple. Firstly, it has a relatively low impact on the overall execution. A code profiled using the execution sampling profiler from Pharo is only 12\% longer than without being profiled. This is reasonable in the large majority of the case developed in Pharo (\ie non-realtime application with little dependencies on the operating system). %[Code1]
Secondly, it has no impact on the profiled application semantics in the large majority of cases. The application is expected to have the same behavior if profiled in the majority of cases. An application that intensively use the language reflective facilities may be perturbed by the regular sampling exercised with the thread launched by the profiler. However, this is hardly the case in practice. 
Thirdly, almost no configuration is required for the profiling to be carried out. Contrary to instrumentation-based profiling, no scope for the instrumentation has to be defined.

In contrast to its numerous advantages, execution sampling remains particularly shy on the analyze one can carry out with it. 

%: % % % % % % % % % % % % % % % % % % % % % % % % % % % % % % % % %
\section{Profiling Metrics}

%: % % % % % % % % % % % % % % % % % % % % % % % % % % % % % % % % %
\section{Related work}\seclabel{relatedwork}

Paper from Hauswirth

%: % % % % % % % % % % % % % % % % % % % % % % % % % % % % % % % % %
\section{Conclusion}\seclabel{conclusion}

%\paragraph{Acknowledments} 

% % % % % % % % % % % % % % % % % % % % % % % % % % % % % % % % % %

%\appendix
%\section{Complete Blueprint}

%{\small
\bibliographystyle{plain}
\bibliography{scg}
%\bibliography{hapao}
%}
\end{document}

%: % % % % % % % % % % % % % % % % % % % % % % % % % % % % % % % % %
